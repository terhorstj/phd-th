% ==========================================================================================
% ==========================================================================================
% ==========================================================================================

% Basic Setup

		\mode<presentation>
				{
					  \usetheme{Madrid}    % Hanover  
					  \usecolortheme{dolphin} % rose
					  \usefonttheme{serif}   % structurebold
					  \setbeamertemplate{navigation symbols}{} 
					  \setbeamertemplate{footline}{}   
					  \setbeamertemplate{frametitle}{\centering\normalsize\bfseries\slshape\insertframetitle\par\vskip-7pt\hrulefill} 
                                          \setbeamercolor{alerted text}{fg=blue}
				} 
               % \addtobeamertemplate{footline}{\vskip-1cm\hskip17pt\insertframenumber\,/\,\inserttotalframenumber\kern1em\vskip2pt}
                
% ==========================================================================================
% ==========================================================================================
% ==========================================================================================

% Packages
             %sansmathfonts

        \usepackage[T1]{fontenc}
		\usepackage{microtype,verbatim,parskip,graphicx,siunitx,nicefrac,booktabs,textpos,mathrsfs} %,rotating,textcomp 
		\usepackage[version=4]{mhchem}
		\usepackage{chemfig}\setchemfig{atom sep=2.5em}

		% Note: sansmathfonts redefines ALL math (including Greek) as sans-serif.
		%       we want this for Greek letters that look nice when mixed with a sans-serif font like helvet.
		%       however, this also forces Arabic numerals into cmu-serif for math mode, which looks bad next to helvet text.
		%       so below, we activate helvet as our main font, then add helvet serif math too.
		%       for some reason, this does not affect the Greek, which is why we need sansmathfonts beforehand.   

		%\usepackage{helvet}
		%\usepackage[helvet]{sfmath}

% ==========================================================================================
% ==========================================================================================
% ==========================================================================================

% Definitions

	    \def\deg{$^{\circ}$}
	    \def\tild{$\sim$}
	    \def\dg{$\Delta G$}
	    \def\electron{e$^-$}
	    \def\Ecell{$\mathscr{E}_{\textnormal{cell}}$}
	    \def\Enotcell{$\mathscr{E}^{\circ}_{\textnormal{cell}}$}
	    \def\Enot{$\mathscr{E}^{\circ}$}
	    \def\not{^\circ}
	    \def\Ka{$K_{\textnormal{a}}$}
	    \def\Ksp{$K_{\textnormal{sp}}$}
	    \def\Keq{$K_{\textnormal{eq}}$}
	    \def\pKa{p$K_{\textnormal{a}}$}
	    \def\dipole{+\hspace*{-3mm}$\longrightarrow$}
	    \def\cmi{cm$^{-1}$}
	    \def\cd{$\cdot$}
	    \def\dd{\end{document}} % useful for debugging

% ==========================================================================================
% ==========================================================================================
% ==========================================================================================

% Commands 

		\newcommand*\unit[1]{\textnormal{ #1}}
		% provides \obox{0}{0}{1cm}{1cm}   where: x y wd ht
		\newcommand\obox[4]{\pgfsetfillopacity{0.5}\begin{textblock}{}(#1,#2)\setlength\textwidth{#3}
		\begin{beamercolorbox}{block body}\vspace*{#4}\end{beamercolorbox}\end{textblock}\pgfsetfillopacity{1}}

		% provides \tbox{0}{0}{text} where: x y text
		\newcommand\tbox[3]{\begin{textblock}{8}(#1,#2)#3\end{textblock}}

% ==========================================================================================
% ==========================================================================================
% ==========================================================================================




















